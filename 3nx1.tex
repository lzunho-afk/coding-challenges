\documentclass{article}

\usepackage{hyperref}
\usepackage{listings}
\usepackage{fancyvrb}

\hypersetup{
  colorlinks=true,
  linkcolor=blue,
  }

\lstset{
  columns=fullflexible,
  mathescape=true,
  literate=
  {=}{$\leftarrow{}$}{1}
  {==}{$={}$}{1},
  morekeywords={if,then,else,return}
  }

\title{O Problema 3n+1}
\date{}

\begin{document}

\maketitle
\begin{center}
O problema (original em inglês) localiza-se \href{https://onlinejudge.org/external/1/100.pdf}{aqui}.
\end{center}

\paragraph{}
Os problemas na Ciência da Computação são frequentemente classificados como pertencendo a determinadas classes.
Nesse problema você vai analisar a propriedade de um algoritmo cuja classificação não é conhecida por todas
as entradas possíveis.
\paragraph{}
Considere o seguinte algoritmo:
\begin{lstlisting}
  input n
  print n
  if n == 1 then STOP
    if n is odd then n = 3n+1
    else n = n/2
  GOTO 2
\end{lstlisting}

\paragraph{}
Dado 22 como entrada, a saída será a seguinte sequencia de números:
\begin{center}
\begin{BVerbatim}
22 11 34 17 52 26 13 40 20 10 5 16 8 4 2 1
\end{BVerbatim}
\end{center}

\paragraph{}
Considera-se que o algoritmo acima irá terminar (1 é escrito) para qualquer entrada com valor integro. Apesar da
simplicidade do algoritmo, não se sabe se esse pressuposto é verdadeiro. Contudo, isso foi verificado para todos
os inteiros \verb|n|, tal qual \verb|0 < n < 1.000.000| (e para mais muitos valores adiante).
\paragraph{}
Dada uma entrada \verb|n|, é possível determinar o número de valores escritos, incluindo o valor 1. Assim, para
determinado \verb|n|, chama-se \verb|tamanho do ciclo| de \verb|n|. No exemplo acima, o tamanho do ciclo de 22 é 16.
\paragraph{}
Para quaisquer dois números \verb|i| e \verb|j|, você deve determinar o tamanho máximo do ciclo de todos os números
entre \verb|i| e \verb|j|, ambos inclusos.

\section*{Entrada}
\paragraph{}
A entrada consiste numa série de pares de números \verb|i,j| (um por linha). Todos os inteiros devem ser menores
que 10.000 e maiores que 0.
\paragraph{}
Você deverá processar todos os pares de inteiros, onde, para cada par, deve-se determinar o tamanho máximo do ciclo
perante todos os inteiros entre \verb|i| e \verb|j|, inclusos.
\paragraph{}
Você pode assumir que nenhuma das operações irá criar um overflow no inteiro de 32 bits.

\section*{Saída}
\paragraph{}
Para cada entrada de pares inteiros \verb|i,j| você deve dar como saída \verb|i|, \verb|j| e o tamanho máximo do ciclo para
os inteiros de \verb|i| até \verb|j|. Esse três números devem ser separados por pelo menos um espaço em branco e
estarem em linha individual (uma linha de entrada por uma de saída). Os inteiros \verb|i| e \verb|j| devem aparecer
na saída na mesma ordem na qual eles aparecem na entrada, seguidos, logo em seguida, pelo tamanho máximo do ciclo (mesma linha).

\section*{Exemplo de entrada}
\begin{BVerbatim}
  1 10
  100 200
  201 210
  900 1000
\end{BVerbatim}

\section*{Exemplo de saída}
\begin{BVerbatim}
  1 10 20
  100 200 125
  201 210 89
  900 1000 174
\end{BVerbatim}

\end{document}
